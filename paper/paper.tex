%Documento LaTeX com o artigo que estamos escrevendo

%Cabeçalho
%Onde configuramos o documento
%%%%%%%%%%%%%%%%%%%%%%%%%%%%%%%%%%%%%%%%%%%%%%%%%%%%%%%%%%%%%%
\documentclass{article}

\usepackage[brazil]{babel}
\usepackage{graphicx}



%Corpo
%Onde escrevesmo o corpo do texto
%%%%%%%%%%%%%%%%%%%%%%%%%%%%%%%%%%%%%%%%%%%%%%%%%%%%%%%%%%%%%%%
\begin{document}
%ambiente é o que está entre o begin e o end

\title{Análise de variação de temperatura dos últimos 5 anos}

\author{Victoria Malta P de Lima}

\maketitle

Meu artigo bem legal.

\begin{abstract}
Resumo babilônico.

\end{abstract}

%formato de um label
%{eq:nome-eq} - se for uma equação
%{fig:nome-fig} - se for figura
\section{Introdução}

Isso vai ser a minha introdução.
Outra frase da introdução

Esse será outro parágrafo da introdução.

\section{Metodologia}

Aqui vou descrever tudo que fiz.
Ajustamos uma reta aos cinco últimos anos dos dados de temperatura média
mensal para cada país.
Assim calculamos uma taxa de variação da temperatura recente.

A equação da reta é

\begin{equation} 
T(t) = a t + b,
\label{eq-reta}
\end{equation}

\noindent %tira a identação
onde $T$ é a temperatura, $t$ minúsculo é o tempo, $a$ é o coeficiente angular e $b$ é o coeficiente linear.

Utilizamos a equação \ref{eq-reta} em código Python para fazer o ajuste da 
reta com o método dos mínimos quadrados.
Isso esta descrito na seção \ref{Metodologia}.

%sifrão demarca o começo da equação que fica no meio do texto
%deve se colocar vírgula depois da equação


\section{Resultados}

Um monte de coisa.

\begin{figure}[tb!] 
	\centering
	\includegraphics[width=05\columnwidth]{../figuras/variacao_temperatura.png}
	\caption{
	Variação de temperatura média anual dos últimos cinco anos:
	a) Países com as cinco maiores variações de temperatura.
	a) Países com as cinco menores variações de temperatura.
	}
	\label{fig:variacao}
\end{figure}

Veja a figura \ref{fi:variacao}.


\end{document}